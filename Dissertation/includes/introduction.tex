\section{Introduction}

With traditional CPUs experiencing slowdowns in improvements relating to
performance, the computing industry is looking for an alternative hardware and
software paradigm to take us into the next era of computing performance. This
has led to the increase in the number of cores on a single CPU die, with most
consumer devices containing four cores allowing four individual threads of
control to run simultaneously on a single chip. In addition to this, the growth
of computational power available in GPUs has led to the idea of using the GPUs
for general purpose computing tasks (GPGPUs). This is known as Heterogeneous
Computing and refers to any system that uses more than one kind of processor. In
addition to this, there are growing numbers of devices with a Heterogeneous
System Architecture (HSA) where multiple processor types (such as CPUs and GPUs)
exist on the same die.

\section{Problem Description}

The primary task of the project was to take document filtering code running on a
CPU and investigate the use of OpenCL to allow the code to be executed on
multiple architectures, for instance a GPU, to improve overall performance of
the system.

Document filtering can place documents into two categories, with the canonical
example being spam or not spam. Documents are given to a scorer as a
collection of terms, with the terms having been generated from a parser. For
each term in a document, it is looked up in a profile of spam terms. Terms in a
document that are typically detected in a spam document will have a score
assigned to them in the profile, it is this score which is collected and added
to the document's overall score. At the end of scoring, if the document has a
high enough score, it is classed as spam, otherwise it is not spam.

\section{Motivation}

Document filtering is problem of growing importance. The advent of the Internet
has resulted in a massive surge in the transfer and storage of data and
information. A prime example is email where hundreds of billions of emails are
sent every day. Emails, and other kinds of documents, sometimes need to be
filtered. This problem is highly data parallel in nature as the classification
of one document does not affect the classification of another. With GPU
manufacturers concentrating on parallelism in their chip designs for pixel
colour value genferation, these devices are thought of as being a suitable device
for other highly parallel tasks.

In addition to the large amounts of data which needs to be filtered, there is a
growing need for large enterprise data centres to be more cost effective, and to
use less energy. A heterogeneous system can not only improve performance in raw
terms, but also improve performance when measured against power and cost of
equipment. These metrics are typically quoted as performance per watt, and
performance per dollar.

\section{Related Research}
\label{sec:relatedResearch}

Document classification is an active area of research, with a number of papers
dedicated to the creation, analysis, and tuning of filtering algorithms. A large
number concentrate on Na{\"{\i}}ve Bayes classification
\cite{androutsopoulos2000evaluation} \cite{androutsopoulos2000learning}  with
different statistical models (for instance multi-variate Bernoulli model or a
multinomial model \cite{Schneider:2003:CEM:1067807.1067848}
\cite{mccallum1998comparison}).

There has also been work looking into the efficiency of different architectures
in the area of information filtering \cite{chen2012invited}
\cite{he2013massively} in general.

This report more specifically relates to, and builds upon, work by Wim
Vanderbauwhede et al. which has primarily focused on the use of Field-
Programmable Gate Arrays (FPGAs) to accelerate a document parser and scoring
system \cite{vanderbauwhede2013high} \cite{HybridCPUFPGA}
\cite{chalamalasetti2012evaluating} with significant success.

\section{Report Structure}

The following sections of the report will detail and discuss the problem of
efficient heterogeneous implementations for document filtering using OpenCL and
will compare results between devices, and to a traditional CPU implementation of
the same algorithm.

Chapter 2 will contain details on the background of OpenCL, the architecture
differences between devices used, and further details of document filtering
using a Na{\"{\i}}ve Bayes classifier.

Chapter 3 will discuss significant implementation details and optimisations of
the document filtering software.

Chapter 4 will evaluate the performance of the document filtering software over
a range of different devices, with a comparison to a traditional CPU
implementation. There will also be a discussion on extra metrics such as
performance per watt and performance per dollar. The chapter will end with an
in-depth discussion of the results.

Chapter 5 will summarise and discuss the results from the evaluation, and will
indicate further work opportunities to increase performance and efficiency
further.
