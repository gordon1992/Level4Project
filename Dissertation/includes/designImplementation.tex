\section{Implementation}

The project implementation is split into two primary sections. The first is the
scoring only implementation, the second is the parsing and scoring
implementation. The scoring only implementation comprised the entire initial
project plan and involved porting the existing scoring code from C++ to OpenCL.
After the implementation of the scoring code, and the initial analysis of
results was conducted, it was determined that the parsing code (also written in
C++) was the bottleneck to the system and the project then began to investigate
also porting the parsing code to OpenCL.

A note on terminology: When a contiguous block of memory is allocated on the
host device it is normally referred to as an array. When the array is
transferred to an OpenCL device it is referred to as a buffer. These are
functionally the same with the primary distinction being where they are
allocated.

\subsection{All implementations}

Most OpenCL applications are implemented in much the same way from the host
code's perspective. There is a significant amount of boiler plate code to detect
OpenCL platforms and devices available on the target system. Each device that is
going to be used needs it's own command queue. An OpenCL command queue is as the
same suggests, any data transfers or kernel executions are placed on the
device's command queue and executed in order of arrival to the queue (although
out of order execution is possible). Command queues, memory, program, and kernel
objects are all managed by an OpenCL context. The kernel itself also has to be
compiled, it is read in from disk as a string and built for the required
devices. This is what allows a single kernel to run on any available device
selected, it's compiled at run time. After compilation is completed, the
kernel's arguments are set to the appropriate device buffers. The host's arrays
that contain data the kernel needs are then transferred to the device's buffers
and the kernel is executed. After the kernel has finished executing, any buffers
relating to results, in this case the document scores, are written back by en-
queuing a read buffer command.

As discussed in the background, the Naive-Bayes classifier makes use of a
profile. This is a 128MB file containing the terms and associated term weights.
This is read in from disk by the C++ host device code, and passed in as a buffer
of unsigned long values to the OpenCL kernel. It does not change between
invocations of the kernel, as all sets of documents are scored using the same
profile.

The use of a bloom filter was also discussed in the background. This is a 4KB
array of bit values used by the OpenCL kernel to see if there is a chance that
the document term being scored exists in the profile. As with the profile, this
is read in from disk and passed in as a buffer to the OpenCL kernel. It also
does not change between invocations of the kernel because the profile does not
change.

OpenCL host code is summarised in the pseudo-code in
Figure~\ref{fig:openCLPseudocode}.

\begin{figure}
\begin{verbatim}
// One time set up
platforms = getPlatforms
devices = getDevices(platforms[0])
context = createContext(devices)
queue = createCommandQueue(context, devices[0])
program = buildProgram(context, sourceCode)
kernel = createKernel(program, kernelMethodName)

// repeat for each buffer
buffer1 = createBuffer(context, options, size)
setArg(kernel, 1, buffer1)
writeBuffer(buffer1)

// Execute kernel and get results back
enqueueKernel(queue, kernel)
readBuffer(resultsBuffer)
\end{verbatim}
\caption{OpenCL Host Pseudo-code}
\label{fig:openCLPseudocode}
\end{figure}

\subsection{Scoring Only}

In addition to the profile and bloom filter, the scoring only implementation
reads in the parsed document terms from disk. These are passed to the OpenCL
kernel and are used to calculate a document's score.

The array of terms is a one dimensional array thus the C++ code has to iterate
over the array and find the beginnings of documents to know how many work items
to create (one per document) and over what range of terms represent each
document. This simply involves the creation of indexes containing the location
of terms with length 0 as this was used to delimit document beginnings. The
array of document marker addresses is prefixed by the number of documents and
has the total number of terms added to the end of the array. The total term
number is required by the work item scoring the last document as there is no
next document to determine where to stop scoring terms at. This buffer is also
transferred to the device.

All buffers are read only with the exception of the scores buffer. This is a
write-only buffer with length equal to the number of documents being scored. It
is this buffer that the kernel writes the results to and is the only buffer that
needs to be written back to the host memory after kernel execution.

After the kernel has the arguments set to the appropriate buffer, and the data
has been transferred to the device the kernel can execute. During kernel
execution, each work item iterates of the terms for their associated document. A
work item knows which document to score because the location of the document
starting term is the work item's ID + 1, up to the work item's ID + 2. Each term
in this range is scored and the total score for the document is stored in a
scores buffer at location equal to the work item's ID.

The OpenCL kernel code is summarised in the pseudo-code in
Figure~\ref{fig:scoringPseudocode}.

\begin{figure}
\begin{verbatim}
documentNumber = get_global_id(0) + 1 // work item ID
startIndex = docAddresses[documentNumber]
endIndex = docAddresses[documentNumber + 1]
score = 0
for (term in terms[startIndex..endIndex]) {
    generateNGrams(term)
    for (ngram in ngrams) {
        address = getTermAddress(ngram)
        profileEntryVector = profile[address]
        for (i = 0; i < 4; i++) {
            profileEntry = profileEntryVector[i]
            if (profileEntry contains ngram) {
                score += getWeight(profileEntry)
            }
        }
    }
}
scores[documentNumber -1] = score
\end{verbatim}
\caption{Scoring Pseudo-code}
\label{fig:scoringPseudocode}
\end{figure}

After the kernel has finished executing, and the scores buffer has been written
to a host device array, the actual classifications of each document is
recovered. A document is classed as significant (spam in the spam or not spam
example) if the score is above a certain threshold.

\subsection{Parsing and Scoring}

For the parsing and scoring version, the host code is very similar. The only
differences relate to the documents themselves. For scoring only, the documents
were already parsed and represented by terms. For parsing and scoring, the plain
text is instead read in. The TREC data collection is read in as a single text
file and passed to the OpenCL kernel as-is. For the document marker array,
instead of detecting zero length terms, the C++ code searches through the plain
text and stores the character locations where the sub-string ``$<$DOC$>$''
begins. This sub-string acts as a direct replacement to the zero length term
locations, and is transferred to the OpenCL kernel in the same way.

Again, each work item is responsible for a single document. During the kernel
invocation, each work items parses the documents character by character and, as
soon as a full term has been parsed, scores the term. A term is ready to score
when a white space character has been reached, and the term is of non-zero
length. Any characters inside tags, for instance the start and end doc tags, are
ignored. After all characters have been parsed, the scores are stored in the
same way as the scoring code.

After the kernel has finished, the scores are written back to the host and the
classifications are calculated as before.

The OpenCL kernel code is summarised in the pseudo-code in
Figure~\ref{fig:parsingScoringPseudocode} where to5BitEncoding(char) is the
Baudot encoding discussed and shown in Figure~\ref{baudotCode}, scoreTerm(term)
contains the profile lookup and scoring addition code from
Figure~\ref{fig:scoringPseudocode}, and updateState(state) works out the parser
state machine transition based on the current state, and what character has just
been found. States tell the parser to either skip characters, write characters
to the term, `flush' the term (aka score the term), or that a tag has been
encountered.

\begin{figure}
\begin{verbatim}
documentNumber = get_global_id(0) + 1 // work item ID
startIndex = docAddresses[documentNumber]
endIndex = docAddresses[documentNumber + 1]
state = SKIPPING
score = 0
for (char in documents[startIndex..endIndex]) {
    if (char is alphanumeric) {
        updateState(state, 0)
        if (state is WRITING) {
            term += to5BitEncoding(char)
        }
    } else if (char is whitespace) {
        updateState(state, 2)
        if (length(term) > 0 and state is FLUSHING) {
            score += scoreTerm(term)
            term = 0 // Parse a new term now
        }
    } else if (char is '<') {
        updateState(state, 3)
    } else if (char is '>') {
        updateState(state, 4)
    } else {
        updateState(state, 1)
    }
}
\end{verbatim}
\caption{Parsing and Scoring Pseudo-code}
\label{fig:parsingScoringPseudocode}
\end{figure}

\section{Optimisations}

\subsection{Local Size}

\subsection{Mapped Buffers}
