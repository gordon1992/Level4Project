\section{OpenCL}

\subsection{Language Overview}

Open Computing Language (OpenCL) is a framework for writing applications
designed to be executed across heterogeneous platforms such as central
processing units (CPUs), graphics processing units (GPUs), and accelerator
devices such as the Intel Phi and Field Programmable Gate Arrays (FPGAs). The
language is an open standard maintained by Khronos Group
\footnote{\url{http://www.khronos.org/opencl}}. OpenCL is based on the C99
standard and can be loosely thought of as a subset of the C99 language with some
additional keywords.

OpenCL splits the program code into two sections called `host' and `device'.
Host code runs on the CPU and is written in a traditional language such as C or
C++. The host code executes as normal and makes use of APIs exposed by OpenCL to
define how, and where, OpenCL code is executed.

OpenCL applications are called kernels. A kernel is a function that executes on
an OpenCL capable device. Given the focus of OpenCL is on parallelism, a kernel
is written from the perspective of a single thread instead of requiring explicit
threading code from more familiar models such as POSIX Threads. An OpenCL work
item is similar to a POSIX Thread with a notable exception that OpenCL has no
stack. OpenCL kernels are also compiled at runtime thus the host code, and the
compiler itself, can make device-specific optimisations for optimal performance.

\subsection{Models}

To accommodate the variety of devices OpenCL is designed to work with, a more
abstract Memory Model, and Concurrency and Execution model was designed to allow
the programmer full access to the target devices' capabilities.

\subsubsection{Memory Model}

For a traditional C/C++ application, the application programmer can make use of
the primary memory space of the computer, typically DDR2 or DDR3 RAM. There are
caches on the CPU die however no direct control over their contents is offered.
For CPUs this is not an issue, all modern CPUs support automatic caching.
Problems occur when considering other devices, such as a GPU. GPUs do not
typically have automatic caching thus a programmer must make use of OpenCLs
abstract Memory Model, and hardware vendors must map this model to the physical
hardware.

The OpenCL memory model defines four different types of memory, each of which
will be outlined below:

Global memory: This memory is accessible by all compute units and is similar to
main memory on a traditional CPU system. For a GPU this is memory is mapped to
the primary memory of the GPU and is typically 1-4GB in size. When transferring
data between the host and device, this is where the data has to reside. Global
memory is denoted as `\_\_global' in an OpenCL kernel.

Constant memory: Constant memory resides in the same space as global memory and
is used for data which is accessed simultaneously by all work items, or for
variables whose values never change. Constant memory is denoted as `\_\_local'
in an OpenCL kernel.

Local memory: This is scratch pad memory which is only visible to a single
compute unit on the device (think core of a CPU). Local memory is generally on-
chip memory and thus has faster access time than global memory. Local memory is
typically on the order of tens of kilobytes in size. Use of local memory is
either statically declared in the kernel (e.g `\_\_local float[64] localArray')
or dynamically allocated at runtime by the host since OpenCL kernels do not
offer dynamic memory allocation. Variables in local memory are not what would be
called `local' in a traditional application, local variables are in private
memory.

Private memory: This is memory that is unique to an individual work item. Any
local variables and non-pointer kernel arguments are private by default. These
are typically mapped to registers although can be slipped over to longer-latency
memories if required.

\subsubsection{Concurrency and Execution Model}

\section{CPU and GPU Differences}

\section{Document Classification}

\subsection{Bloom Filters}

\subsection{Hashed Profile}

% linear probing, hence 4 terms looked at.
