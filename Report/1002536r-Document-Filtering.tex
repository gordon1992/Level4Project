\pdfoutput=1
\documentclass[11pt]{article}
\usepackage{geometry}
\geometry{a4paper}
\usepackage[utf8]{inputenc}
\usepackage[parfill]{parskip}
\usepackage{sectsty}
\usepackage{fullpage}
\begin{document}
\title{GPU-Accelerated Document Filtering}
\author{Gordon Reid - 1002536r}
\date{December 20, 2013}
\maketitle

\section*{Project Description}

With traditional CPUs experiencing slowdowns in improvements with clock speeds,
the computing industry is looking for an alternative hardware and software
paradigm to take us into the next era of computing performance. This has led to
the increase in the number of cores on a single CPU die, with most commercial
devices containing four cores allowing four individual threads of control to
run simultaneously on a single chip. In addition to this the growth of
computational power available in GPUs has led to the idea of using the GPUs for
general purpose computing tasks (GPGPUs).

The primary aim of the project is to take document filtering running on a CPU
and investigate the use of a GPU with OpenCL to improve the overall performance
of the system. This will begin with porting the scoring section of the document
classifier from the CPU to the GPU and comparing performance with the all CPU
version and the CPU and FPGA version. Further performance tests can be
conducted by employing a bloom filter which is a small data structure which can
be used to reduce the number of requests for memory in slow, global memory by
sitting in faster, local memory and returning a bit value stating whether or
not a given item is in global memory or not.

Document filtering can place documents into two categories, with the canonical
example being spam or not spam. Documents are given to the scorer as a
collection of terms. For each term in a document, it is looked up in a profile
of spam terms. Terms in a document that are typically detected in a spam
document will have a score assigned to them in the profile, it is this score
which is collected and added to the document's overall score. At the end of
scoring, if the document has a high enough score, it is classed as spam,
otherwise it is not spam.

\section*{Progress}

Currently code has been written for the scoring of documents in a single CPU
thread, multiple CPU threads, and an OpenCL implementation for the CPU or GPU.
Performance testing has been conducted on all four implementations, each over
twelve profiles and four different forms of bloom filter: None, All Zero, All
One, and Realistic. The different bloom filters has allowed conclusions
regarding the range of performance metrics possible with different bloom filter
occupancies to be made. Results for measuring the total transfer time between
the host and device have been evaluated which has allowed for conclusions
regarding the effect of PCI-E Bus Speed to be made.

\section*{Plan}

The code I have written could be integrated within the existing parser system,
to allow for full system performance metrics to be calculated. This will also
allow for an extension to the OpenCL code base with the inclusion of the
parsing stage. This will allow the investigation into how important accelerator
chips such as GPUs are in improving overall system performance, rather than the
individual task of scoring.

\section*{Problems}

I have experienced a few issues so far in the project, the primary issues
revolved around OpenCL and the corresponding SDKs. My desktop machine has an
Intel i5-2500K which is OpenCL capable however installing the Intel SDK has
proved troublesome and thus far the CPU has only been used for the traditional
C++ code. This was rectified by the use of a University's Mac Pro which had an
OpenCL supporting CPU and GPU allowing benchmark runs all on the one machine.

The OpenCL language and behaviour has been somewhat troublesome as well. OpenCL
is roughly a subset of the C99 standard language however compiler errors and
runtime error behaviour is significantly different from standard C/C++
applications. An example of this was an out of bounds error early on in the
project, instead of receiving a Segmentation Fault during kernel execution, an
error was being thrown afterwards, when the result of the kernel was being
written back to the host CPU. This unfamiliar behaviour led to a slower than
expected debug process.

\end{document}
